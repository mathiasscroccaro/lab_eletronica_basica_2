\documentclass{article}[12pt]
\usepackage[utf8]{inputenc}
\usepackage{float}
\usepackage[portuguese]{babel}
\usepackage{graphicx}
\usepackage{subfigure}
\usepackage{amsmath}

%%% --- HIPERLINKS --- %%%

\usepackage[colorlinks=true,plainpages=true,citecolor=black,linkcolor=black, urlcolor=blue]{hyperref}

\usepackage{indentfirst}
\usepackage{tabularx}
\usepackage{multirow}
\usepackage[a4paper,
 top=3cm, 
 bottom=2cm, 
 left=3cm, 
 right=2cm]{geometry} 
 
\usepackage{titlesec}
\usepackage{helvet}

%%% --- CABEÇALHO E RODAPÉ --- %%%

\usepackage{fancyhdr}
\pagestyle{fancy}
\fancyhf{}

\fancyhead[L]{\footnotesize Laboratório de Eletrônica Básica II - EE641}
\fancyhead[R]{\footnotesize Plano Didático}   
\fancyfoot[R]{\footnotesize 2020} 
\fancyfoot[C]{} 
\fancyfoot[L]{\footnotesize Faculdade de Engenharia Elétrica e Computação - Unicamp} 
\renewcommand{\footrulewidth}{0.7pt}
\renewcommand{\headrulewidth}{0.7pt}
 

%\renewcommand{\familydefault}{\sfdefault}
\renewcommand{\baselinestretch}{1.5} 
\titlespacing{\section}{0pt}{36pt}{24pt}
\setlength{\parindent}{4em}

\def \proporcaoFigura {0.8}

\begin{document}


\newpage

\normalsize

\section*{Plano Didático\\EE641 - Laboratório de Eletrônica Básica II}

\subsection*{Informações:}

\textbf{Professores:} Dr. Eduardo Tavares Costa e Dr. Leandro Tiago Manera

\textbf{PED:} Mathias Scroccaro Costa \hfill mathias.scroccaro@gmail.com

\subsection*{Conteúdo programático:}

Os alunos serão instruidos à montagem em placa perfurada padrão de um circuito didático para simulação, amplificação e processamento de sinais de ECG (Eletrocardiograma). Serão empregados circuitos integrados reguladores lineares de tensão, amplificadores operacionais, redes de realimentação e divisores resistivos. A plataforma Arduino será utilizada como interface de controle, a qual deverá ser programada. Desta maneira, os seguintes conteúdos específicos serão abordados:

\begin{itemize}
\item Realimentação e reguladores lineares de tensão;
\item Topologias \textit{buffer}, inversor, somador e subtrator; 
\item Amplificadores de instrumentação;
\item Filtros ativos de primeira e segunda ordem (topologia Sallen-Key);
\item Linguagem de programação C ou C++;
\end{itemize}

\subsection*{Programação das atividades:}

\begin{table}[!h]
\centering
\begin{tabular}{|c|c|l|}
\hline
\textbf{Encontro} & \textbf{Data} & \textbf{Conteúdo} \\ \hline
\#1 & 20/01 (Segunda-feira) & \begin{tabular}[c]{@{}l@{}}Apresentação\\Realimentação e reguladores de tensão\end{tabular} \\ \hline
\#2 & 21/01 (Terça-feira) & DAC1 - Somador 8 \textit{bits} e filtro 1ª ordem \\ \hline
\#3 & 23/01 (Quinta-feira) & DAC2 - Adição de ruído e correção de \textit{offset} \\ \hline
\#4 & 24/01 (Sexta-feira) & Amplificador de instrumentação pt. 1 \\ \hline
\#5 & 27/01 (Segunda-feira) & Amplificador de instrumentação pt. 2\\ \hline
\#6 & 28/01 (Terça-feira) & Filtro 2ª ordem Sallen-key \\ \hline
\#7 & 30/01 (Quinta-feira) & Apresentação do projeto final \\ \hline
\end{tabular}
\end{table}

\pagebreak

\subsection*{Orientações para uso dos laboratórios:}

\begin{itemize}
\item Proibido comer ou ingerir líquidos;
\item Proibido uso de calçados abertos;
\item Manter ambiente organizado;
\item Após utilizar cabos e instrumentos, reorganizá-los no devido local de origem.
\end{itemize}

\subsection*{Avaliação:}

A avaliação será composta pela soma ponderada dos relatórios de experimento (RE) e apresentação do projeto final (PF), de maneira que
\begin{equation*}
\textrm{Nota Final} ~=~ 0,4 * \textrm{RE} ~+~ 0,6 * \textrm{PF}.
\end{equation*}

\nocite{sedra2007microeletronica}

% \begin{figure}[hbt]
% \centering
%   \includegraphics[width=10cm]{}
%   \caption{}
%   \label{}
% \end{figure}

\bibliography{bibli}
\bibliographystyle{ieeetr}





\end{document}
